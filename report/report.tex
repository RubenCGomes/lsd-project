% Preamble
\documentclass{report}

% Packages
\usepackage{amsmath}
\usepackage[T1]{fontenc} % Fontes T1
\usepackage[utf8]{inputenc} % Input UTF8
\usepackage[backend=biber, style=ieee]{biblatex} % para usar bibliografia
\usepackage{csquotes}
\usepackage[portuguese]{babel} %Usar língua portuguesa
\usepackage{blindtext}
\usepackage{graphicx} % Gerar texto automaticamente
\usepackage{geometry}

% Document
\begin{document}
    % Define keybinds
    \def\title{\textbf{MÁQUINA DE FAZER PÃO}}
    \def\authors{João Bastos (113470), Rúben Gomes(113435)}
    \def\contacts{(113470) joaop.bastos@ua.pt, (113435) rlcg@ua.pt}
    \def\department{Departamento de Eletrónica, Telecomunicações e Informática}
    \def\university{Universidade de Aveiro}
    \def\logo{ua.pdf}


    % Header
    \fancyhead{
        \begin{center}
            \includegraphics{\logo}\\
            \textbf{{\LARGE \title}}\\
            \vspace{2mm}
            \Large Trabalho realizado por: \authors.\\
        \end{center}
    }

    % Introduction
    \section{Introdução}\label{sec:introducao}
    Este projeto visa a implementar uma máquina de fazer pão, que permita ao utilizador escolher o tipo de pão que pretende fazer. \ Para tal, é necessário que o programa tenha uma máquina de estados finitos que permita ao utilizador escolher o tipo de pão a ser fabricado facilmente, bem como adicionar tempo extra de cozedura e/ou tempo prévio de espera para iniciar a máquina.\\
    \par Para tornar este projeto possível, pretendemos implementar uma máquina de estados, e os seus componentes necessários em linguagem de \textit{hardware} \ac{VHDL}, de forma a aplicar a matéria lecionada em Laboratório de Sistemas Digitais(LSD)





    \begin{flushright}
        \today
    \end{flushright}
\end{document}